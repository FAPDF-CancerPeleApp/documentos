% \part{Related Works}

\chapter[Related Works]{Related Works}


\section{In Search of Big Medical Data Integration Solutions - A Comprehensive Survey}
A integração de dados médicos, especialmente em oncologia, enfrenta desafios devido à heterogeneidade e complexidade dos dados, 
incluindo imagens médicas, informações genômicas e registros eletrônicos de saúde (EHRs). Abordagens como data lakes e data 
warehouses modernos têm se destacado na consolidação de grandes volumes de dados estruturados e não estruturados, possibilitando o
 armazenamento e processamento eficiente de informações diversas, como imagens de ressonância magnética e dados genômicos. Ferramentas 
 baseadas em Hadoop, Apache Hive e frameworks semelhantes são amplamente utilizadas para criar plataformas escaláveis e integradas, 
 promovendo análises mais eficazes, como na previsão de doenças crônicas e prevenção de condições como obesidade. \cite{dhayne2019}

Adicionalmente, soluções baseadas em web semântica e aprendizado de máquina vêm contribuindo significativamente para melhorar a 
interoperabilidade e a análise de dados médicos. Tecnologias como Linked Open Data (LOD) e ontologias médicas (SNOMED CT e UMLS) 
aprimoram a semântica dos dados, enquanto modelos de aprendizado profundo (deep learning) avançam na análise de imagens médicas, 
auxiliando na detecção precoce de doenças e personalização de tratamentos. Iniciativas como o iManageCancer exemplificam o uso de 
data warehouses semânticos para integrar e analisar dados de forma avançada, reforçando o potencial dessas abordagens no apoio à 
saúde pública e à medicina personalizada. \cite{dhayne2019}


\section{What Clinics Are Expecting From Data Scientists? A Review on Data Oriented Studies Through Qualitative and Quantitative Approaches}
A integração de análise de dados em ambientes clínicos, sob o paradigma de "Connected Health" (CH), está transformando a forma como dados 
médicos são utilizados para melhorar a tomada de decisão e o manejo de doenças. A "Translational Medicine" (TM) atua como um elo essencial 
entre cientistas de dados e profissionais de saúde, empregando análises preditivas de prontuários eletrônicos e resultados laboratoriais 
para identificar riscos e aprimorar diagnósticos. Embora abordagens quantitativas sejam predominantes, a combinação com técnicas qualitativas 
tem demonstrado maior eficácia em áreas como a interpretação de narrativas de pacientes e análise de padrões complexos, especialmente em 
contextos de doenças como câncer, onde fontes de dados variados, como imagens médicas e genômicas, precisam ser integradas. \cite{xu2019clinics}

Apesar dos avanços, desafios éticos e técnicos, como interoperabilidade e privacidade, permanecem obstáculos significativos. Padrões como HL7
 e iniciativas como "Big Data to Knowledge" (BD2K) têm promovido o uso de big data biomédico para superar essas barreiras. A integração de 
 abordagens qualitativas e quantitativas na análise de dados de câncer, aliada à interoperabilidade, representa uma oportunidade crucial para 
 melhorar diagnósticos precoces e tratamentos personalizados. O desenvolvimento de arquiteturas de dados robustas e interdisciplinares continua 
 sendo um foco promissor para avanços em sistemas de saúde conectados. \cite{xu2019clinics}