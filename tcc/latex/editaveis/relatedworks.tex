% \part{Related Works}

\chapter[Related Works]{Related Works}


\section{In Search of Big Medical Data Integration Solutions - A Comprehensive Survey}

A integração de dados médicos, especialmente em domínios como a oncologia, tem sido foco de extensa pesquisa devido ao crescimento 
exponencial de dados heterogêneos, incluindo imagens médicas, dados genômicos e registros de saúde eletrônicos (EHRs). A literatura 
destaca múltiplas abordagens para enfrentar os desafios de integração e análise de dados no setor de saúde. \cite{dhayne2019}

Dhayne et al. \cite{dhayne2019} revisam tecnologias e ferramentas voltadas para a integração de dados médicos, identificando desafios relacionados à 
heterogeneidade de fontes e complexidade semântica dos dados. Técnicas como data warehouses modernos, data lakes e ferramentas baseadas 
em Hadoop têm sido propostas para consolidar e gerenciar grandes volumes de dados médicos estruturados e não estruturados. Por exemplo, 
o uso de arquiteturas baseadas em data lakes demonstrou potencial para integrar dados diversos, como imagens de ressonância magnética e 
dados de genômica, simplificando o armazenamento e o processamento.

Além disso, soluções baseadas em web semântica e aprendizado de máquina têm se mostrado promissoras. Tecnologias como o Linked Open Data (LOD) 
e ontologias médicas (ex.: SNOMED CT e UMLS) foram aplicadas para melhorar a interoperabilidade e a análise semântica de dados complexos. 
Por outro lado, modelos de aprendizado profundo (deep learning) têm sido amplamente empregados para análise de imagens médicas, promovendo 
a detecção precoce e a personalização de tratamentos em oncologia. \cite{dhayne2019}

Projetos recentes, como o iManageCancer, demonstram a aplicação de data warehouses semânticos para integrar dados estruturados e não estruturados, 
facilitando análises avançadas e previsões em saúde pública. Iniciativas semelhantes têm usado frameworks como Apache Hive e Hadoop para criar 
plataformas escaláveis de integração e processamento de dados, destacando-se em domínios como prevenção de obesidade e previsão de doenças crônicas. \cite{dhayne2019}

Apesar do progresso significativo, a literatura aponta lacunas em relação à integração de dados médicos baseados em imagens, como a ausência de 
abordagens padronizadas para integrar metadados e dados visuais em um framework unificado. Este trabalho busca abordar essa lacuna, propondo uma 
arquitetura de data fabric específica para imagens de câncer, com foco na interoperabilidade e eficiência analítica.


\section{What Clinics Are Expecting From Data Scientists? A Review on Data Oriented Studies Through Qualitative and Quantitative Approaches}
 
Recentemente, a exploração de como a análise de dados pode ser integrada a ambientes clínicos para melhorar a tomada de decisão médica e 
o manejo de doenças. A ideia de "Connected Health" (CH), que conecta pacientes, clínicos e pesquisadores, vem sendo amplamente discutida como 
um paradigma para transformar dados médicos em informações úteis e acionáveis. Em particular, a "Translational Medicine" (TM) atua como uma 
ponte entre cientistas de dados e profissionais de saúde, visando à análise preditiva de dados como prontuários eletrônicos e resultados 
laboratoriais para identificar riscos e formular diagnósticos mais precisos. \cite{xu2019clinics}

No entanto, enquanto a análise quantitativa é dominante na pesquisa de saúde, revisões apontam que abordagens qualitativas têm demonstrado 
maior aplicabilidade em áreas como a interpretação de narrativas de pacientes e a análise de padrões complexos em dados de saúde. Estudos 
demonstraram que técnicas qualitativas, combinadas com análise quantitativa, permitem insights mais abrangentes, especialmente no contexto 
de doenças como câncer, onde múltiplas fontes de dados, como imagens médicas e genômicas, precisam ser integradas para melhores resultados 
clínicos. \cite{xu2019clinics}

Ainda assim, desafios éticos e técnicos relacionados à interoperabilidade de dados e à preservação de privacidade continuam sendo barreiras 
significativas para o uso pleno de sistemas como registros eletrônicos de saúde (EHR). Modelos de sucesso incluem a adoção de padrões de 
interoperabilidade como o HL7 e iniciativas como o programa "Big Data to Knowledge" (BD2K), que promove a utilização de big data para extrair 
informações significativas de dados biomédicos.

A integração de técnicas qualitativas e quantitativas na análise de dados de imagens médicas em câncer, alinhada a estratégias de interoperabilidade 
e padronização, representa uma oportunidade crucial para melhorar a detecção precoce e os tratamentos personalizados. Assim, o desenvolvimento de 
arquiteturas de dados robustas e interdisciplinares permanece um campo de interesse crescente, com potencial para avanços significativos em 
sistemas de saúde conectados. \cite{xu2019clinics}