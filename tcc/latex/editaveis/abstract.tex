\begin{resumo}[Abstract]
 \begin{otherlanguage*}{english}
    This work proposes a Data Fabric architecture for storing and processing skin cancer images, aiming to facilitate the efficient, secure, and scalable management of such data. The study is based on a theoretical and conceptual approach, using a literature review to support the feasibility of the proposed solution. The methodology includes a case study that analyzes the challenges, requirements, and possible solutions for adopting a Data Fabric infrastructure in this context. The proposed architecture is structured to ensure data integration from multiple sources, security and privacy in image storage and sharing, as well as enabling centralized governance and system scalability. Based on the literature review and theoretical study, it is observed that Data Fabric has the potential to optimize access and processing of medical images, promoting interoperability and efficiency in data management. As a conclusion, the study proposes a roadmap for future implementation of the architecture, structured in sequential stages: setting up the development environment and deploying the Kubernetes cluster, integrating MinIO for distributed data storage, installing Delta Lake for transactional data management, configuring Apache Airflow for pipeline orchestration, and finally, implementing Delta Sharing for secure data sharing. The conclusion reinforces the importance of continuing research, highlighting the need for practical validation of the architecture and the investigation of technical and regulatory challenges for its implementation in real clinical environments.

   \vspace{\onelineskip}
 
   \noindent 
   \textbf{Key-words}: Data Fabric. Data Architecture. Medical Image Storage. Information Security. Interoperability.
 \end{otherlanguage*}
\end{resumo}
