\begin{resumo}
    Este trabalho propõe uma arquitetura de Data Fabric para armazenamento e processamento de imagens de câncer de pele, visando facilitar a gestão eficiente, segura e escalável desses dados. O estudo se baseia em uma abordagem teórica e conceitual, e utiliza revisão bibliográfica para embasar a viabilidade da proposta. A metodologia inclui um estudo de caso que analisa os desafios, requisitos e possíveis soluções para a adoção de uma infraestrutura de Data Fabric nesse contexto. A arquitetura proposta é estruturada para garantir integração de dados de diversas fontes, segurança e privacidade no armazenamento e compartilhamento das imagens, além de possibilitar a governança centralizada e a escalabilidade do sistema. Com base na revisão bibliográfica e no estudo teórico, observa-se que o Data Fabric apresenta potencial para otimizar o acesso e processamento de imagens médicas, promovendo interoperabilidade e eficiência no tratamento de dados. Como conclusão, o estudo propõe um roadmap para implementação futura da arquitetura, estruturado em etapas sequenciais: configuração do ambiente de desenvolvimento e implantação do cluster Kubernetes, integração do MinIO para armazenamento distribuído, instalação do Delta Lake para gestão transacional de dados, configuração do Apache Airflow para orquestração de pipelines e, por fim, implementação do Delta Sharing para compartilhamento seguro de dados. A conclusão reforça a importância da continuidade da pesquisa, destacando a necessidade de validação prática da arquitetura e a investigação de desafios técnicos e regulatórios para sua implementação em ambientes clínicos reais.
 \vspace{\onelineskip}
    
 \noindent
 \textbf{Palavras-chave}: Data Fabric. Arquitetura de Dados. Armazenamento de Imagens Médicas. Segurança da Informação. Interoperabilidade.
\end{resumo}
