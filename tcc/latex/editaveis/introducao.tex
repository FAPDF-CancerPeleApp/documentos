\chapter[Introdução]{Introdução}
% \addcontentsline{toc}{chapter}{Introdução}

\section{INTRODUÇÃO}

Com o avanço da tecnologia, armazenar e processar grandes volumes de dados ficou mais ágil e inteligente. Nesse cenário, o Data Fabric se destaca como uma solução poderosa, especialmente para lidar com imagens dos diferentes fontes. Ele cria uma conexão fluida e integrada entre os dados, permitindo acesso rápido, seguro e escalável. 

Este trabalho tem como objetivo propor uma arquitetura de Data Fabric para armazenar e processar imagens de câncer de pele. A metodologia usada inclui uma revisão detalhada da literatura já existente e um estudo de caso que analisa os desafios, requisitos e possíveis soluções para implementar essa infraestrutura. A revisão da literatura fornece a base teórica necessária para a viabilidade da proposta, enquanto o estudo de caso oferece uma análise prática das questões envolvidas.

A arquitetura proposta neste estudo é projetada para integrar dados de várias fontes, garantindo a segurança e privacidade no armazenamento e compartilhamento das imagens. Além disso, ela permite uma governança centralizada e a escalabilidade do sistema, aspectos fundamentais para a gestão eficiente de dados médicos. A integração de ferramentas como Kubernetes, MinIO, Delta Lake, Apache Airflow e Delta Sharing é detalhada, destacando suas funções e benefícios no contexto da arquitetura de Data Fabric.



\subsection{Justificativa}

A demanda por soluções mais inteligentes e seguras para armazenar e processar imagens cresce a cada dia, impulsionando a criação de novas arquiteturas de dados. O Data Fabric surge como uma resposta moderna a esse desafio, oferecendo uma gestão integrada, escalável e eficiente. Com foco na privacidade e na segurança, essa abordagem garante que as informações sejam acessadas e utilizadas de forma confiável, simplificando processos.


\subsection{Objetivos}

Os objetivos deste trabalho são:
\begin{itemize}
    \item Propor uma arquitetura de Data Fabric para armazenar e processar imagens de câncer de pele.
    \item Analisar os desafios e requisitos para a implementação desta arquitetura.
    \item Apresentar um roadmap detalhado para a implementação da arquitetura proposta.
    \item Unifica dados de várias fontes em um armazenamento MinIO escalável, permitindo análises avançadas com segurança e eficiência.
\end{itemize}

\subsection{Estrutura do Documento}

Este documento está estruturado da seguinte forma:
\begin{itemize}
    \item \textbf{Capítulo 1} - Introdução: Apresenta o contexto, a justificativa, os objetivos e a estrutura do documento.
    \item \textbf{Capítulo 2} - Referencial Teórico: Revisão da literatura sobre Data Fabric, Engenharia de Dados, Data Warehouse, Data Lake, Data Mesh, Apache Airflow, Apache Spark, Kubernetes, FastAPI, e Pipelines de dados.
    \item \textbf{Capítulo 3} - Metodologia: Detalhamento da metodologia adotada, incluindo o estudo de caso e as perguntas de pesquisa.
    \item \textbf{Capítulo 4} - Projeto: Descrição dos requisitos, arquitetura proposta, tecnologias utilizadas e roadmap de implementação.
\end{itemize}